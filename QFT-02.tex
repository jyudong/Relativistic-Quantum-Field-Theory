% !TEX program = pdflatex
% !TEX options = -synctex=1 -interaction=nonstopmode -file-line-error "%DOC%"
% 作业模板
\documentclass[UTF8,10pt,a4paper]{article}
\usepackage[UTF8]{ctex}% 中文支持
\usepackage{braket}% Q.M. 
\usepackage{enumerate}
% 将全角句号映射为全角句点,需xelatex编译
% \catcode`\。=\active
% \newcommand{。}{.}
% 作业信息
\newcommand{\CourseName}{相对论量子场论}
\newcommand{\CourseCode}{PHYS2124}
\newcommand{\Semester}{22-23学年秋学期}
\newcommand{\ProjectName}{作业二}
\newcommand{\DueTimeType}{截止日期}
\newcommand{\DueTime}{22.09.27}
\newcommand{\StudentName}{董建宇}
\newcommand{\StudentID}{2019511017}
\usepackage[vmargin=1in,hmargin=.5in]{geometry}% 页边距
\usepackage{fancyhdr}% 页眉和页脚
\usepackage{lastpage}% 返回末页页码
\usepackage{calc}% 返回文本宽度
\pagestyle{fancy}% 全局页眉和页脚风格
\fancyhf{}% 清除预设的页眉和页脚
\fancyhead[L]{\CourseName}% 页眉左
\fancyhead[C]{\ProjectName}% 页眉中
\fancyhead[R]{\StudentName}% 页眉右
\fancyfoot[R]{\thepage\ / \pageref{LastPage}}% 页脚右
\setlength\headheight{12pt}% 页眉高
\fancypagestyle{FirstPageStyle}{% 首页页眉和页脚风格
    \fancyhf{}% 清除预设的页眉和页脚
    \fancyhead[L]{\CourseName\\
        \CourseCode\\
        \Semester}% 页眉左
    \fancyhead[C]{{\Huge\bfseries\ProjectName}\\
        \DueTimeType\ : \DueTime}% 页眉中
    \fancyhead[R]{姓名 : \makebox[\widthof{\StudentID}][s]{\StudentName}\\
        学号 : \StudentID\\
        成绩 : \underline{\makebox[\widthof{\StudentID}]{}}}% 页眉右
    \fancyfoot[R]{\thepage\ / \pageref{LastPage}}% 页脚右
    \setlength\headheight{36pt}% 页眉高
}
\usepackage{amsmath,amssymb,amsthm,bm}% 基础数学支持,特殊数学字符,自定义定理,公式内加粗
\allowdisplaybreaks[4]% 公式跨页
\newtheoremstyle{Problem}% 定理风格名称
{}% 定理上方空间尺寸,留空为默认
{}% 定理下方空间尺寸,留空为默认
{}% 定理主体字体
{}% 定理缩进量
{\bfseries}% 定理开头字体
{.}% 定理开头后所接标点
{ }% 定理开头后所接空间尺寸,空格为默认词间距
{第\thmnumber{ #2}\thmname{ #1}\thmnote{ (#3)} 得分: \underline{\qquad\qquad}}% 定理开头格式,留空为默认
\theoremstyle{Problem}% 设定定理风格
\newtheorem{prob}{题}% 题目
\newtheoremstyle{Solution}% 定理风格名称
{}% 定理上方空间尺寸,留空为默认
{}% 定理下方空间尺寸,留空为默认
{}% 定理主体文本字体
{}% 定理缩进量
{\bfseries}% 定理开头字体
{:}% 定理开头后所接标点
{ }% 定理开头后所接空间尺寸,空格为默认词间距
{\thmname{#1}}% 定理开头格式
\makeatletter
\def\@endtheorem{\qed\endtrivlist\@endpefalse}% 题解后添加qed符号(黑色空心小正方形)
\makeatother
\theoremstyle{Solution}% 设定定理风格
\newtheorem*{sol}{解}% 题解
% \usepackage{mathrsfs}% 公式内花体字母 - \mathscr{}
% \usepackage{esint}% 特殊积分号
% \providecommand{\abs}[1]{\left\lvert#1\right\rvert}% 绝对值 - \abs{}
% \providecommand{\norm}[1]{\left\lVert#1\right\rVert}% 范数 - \norm{}
% \providecommand{\bra}[1]{\left\langle#1\right\rvert}% 左矢 - \bra{}
% \providecommand{\ket}[1]{\left\lvert#1\right\rangle}% 右矢 - \ket{}
% \providecommand{\braket}[2]{\left\langle#1\vert#2\right\rangle}% 右矢接左矢 - \braket{}{}
% \usepackage{graphicx}% 图片 -
% \begin{figure}[htbp]% 图片位置优先顺序:当地,页顶,页底,另起一页
%     \centering% 图片居中
%     \includegraphics[scale=.5]{图片文件名/路径名}
%     \caption{图片文字说明}
%     \label{图片引用代码}
% \end{figure}
% \usepackage{float}% 强制设定图片位置 - [H]
% \usepackage{subfigure}% figure环境内多子图 -
% \begin{figure}[htbp]% figure环境位置优先顺序:当地、页顶、页底、另起一页
%     \centering% figure环境居中
%     \subfigure[子图文字说明]{
%         \label{子图引用代码}
%         \includegraphics[width=0.45\textwidth]{子图文件名/路径}}
%     \subfigure[子图文字说明]{
%         \label{子图引用代码}
%         \includegraphics[width=0.45\textwidth]{子图文件名/路径}}
%     \caption{总文字说明}
%     \label{总引用代码}
% \end{figure}
% \usepackage{multirow}% 表格内多行单元格合并
% \usepackage{booktabs}% 三线表 - \toprule, \midrule, \bottomrule
% \usepackage{longtable}% 表格跨页 -
% \begin{center}% 表格居中
%     \begin{longtable}{lcr}% 表格列对齐: l = 居左, c = 居中, r = 居右
%         \caption{表格文字说明}
%         \label{表格引用代码}
%     \end{longtable}
% \end{center}
% \usepackage[version=4]{mhchem}% 化学式 - \ce{}


\begin{document}
\thispagestyle{FirstPageStyle}% 设定首页页眉和页脚风格
\begin{prob}[]证明自由Hermitian标量场的对易关系
    \[
        [\varphi(\vec{r}_1, t_1), \varphi(\vec{r}_2, t_2)] = -i D(x_1-x_2)
    \]
    可以写成下列形式:
    \[
        D(x_1-x_2) = \int \frac{d^3\vec{k}}{(2\pi)^3} e^{i\vec{k}\cdot(\vec{r}_1-\vec{r}_2)} \frac{\sin\omega(t_1-t_2)}{\omega}
    \]
    \[
        \omega = \sqrt{\vec{k}^2 + m^2}
    \]
    而且$D(x)$满足下列偏微分方程:
    \[
        \left( -\frac{\partial^2}{\partial t^2} + \nabla^2 - m^2 \right) D(x) = 0
    \]
    和当$t=0$时的初始条件:
    \[
        D(x) = 0, \dot{D}(x) = \delta^3(\vec{r})
    \]
    其中$x_1 = (\vec{r}_1, it_1), x_2 = (\vec{r}_2, it_2), x = (\vec{r}, it)$

\end{prob}

\begin{sol}
    由Fourier展开可以得:
    \[
        \varphi(\vec{r}, t) = \sum_{\vec{k}} \frac{1}{\sqrt{2\omega\Omega}} \left( a_{\vec{k}} e^{i\vec{k}\cdot\vec{r} - i\omega t} + a_{\vec{k}}^\dagger e^{-i\vec{k} \cdot \vec{r} + i\omega t} \right).
    \]
    则可以计算
    \begin{align*}
        [\varphi(\vec{r}_1, t_1), \varphi(\vec{r}_2, t_2)] =& \frac{1}{2\Omega} \sum_{\vec{k},\vec{k}'} \frac{1}{\sqrt{\omega\omega'}} [a_{\vec{k}}e^{i\vec{k}\cdot\vec{r}_1 - i\omega t_1} + a_{\vec{k}}^\dagger e^{-i\vec{k}\cdot\vec{r}_1 + i\omega t_1}, a_{\vec{k}'}e^{i\vec{k}'\cdot\vec{r}_2 - i\omega t_2} + a_{\vec{k}{'}}^\dagger e^{-i\vec{k}'\cdot\vec{r}_2 + i\omega t_2}] \\
        =& \frac{1}{2\Omega} \sum_{\vec{k}} \frac{1}{\omega} \left( e^{i\vec{k} \cdot (\vec{r}_1 - \vec{r}_2) - i\omega(t_1-t_2)} -  e^{-i\vec{k}(\vec{r}_1 - \vec{r}_2) + i\omega (t_1-t_2)} \right) \\
        =& \frac{1}{2\Omega} \sum_{\vec{k}} \frac{1}{\omega} \left( e^{i(\vec{k}\cdot\vec{r} - \omega t)} - e^{-i(\vec{k}\cdot\vec{r} - \omega t)} \right) \\
        =& \sum_{\vec{k}} \frac{i}{\Omega} \frac{\sin(\vec{k} \cdot \vec{r} - \omega t)}{\omega} \\
        =& i \int \frac{d^3\vec{k}}{(2\pi)^3} \frac{\sin(\vec{k} \cdot \vec{r}) \cos(\omega t) - \cos(\vec{k} \cdot \vec{r}) \sin(\omega t)}{\omega}
    \end{align*}
    由于$\frac{\sin(\vec{k} \cdot \vec{r}) \cos(\omega t)}{\omega}$关于$\vec{k}$是奇函数,所以体积分为零,因此上述对易子可进一步计算得:
    \begin{align*}
        [\varphi(\vec{r}_1, t_1), \varphi(\vec{r}_2, t_2)] =& -i\int \frac{d^3\vec{k}}{(2\pi)^3} \frac{\cos(\vec{k} \cdot \vec{r})\sin(\omega t)}{\omega} \\
        =& -i\int \frac{d^3\vec{k}}{(2\pi)^3} \frac{(\cos(\vec{k}\cdot\vec{r}) + i\sin(\vec{k}\cdot\vec{r})) \sin(\omega t)}{\omega} \\
        =& -i \int \frac{d^3\vec{k}}{(2\pi)^3} \frac{e^{i\vec{k} \cdot \vec{r}} \sin(\omega t)}{\omega} \\
        =& -iD(x_1 - x_2)
    \end{align*}
    则有
    \[
        D(x_1 - x_2) = \int \frac{d^3\vec{k}}{(2\pi)^3}e^{i\vec{k}\cdot(\vec{r}_1 - \vec{r}_2)} \frac{\sin\omega(t_1-t_2)}{\omega}
    \]
    可以验证:
    \begin{align*}
        \left( -\frac{\partial^2}{\partial t^2} + \nabla^2 - m^2 \right)D(x) =& \int \frac{d^3\vec{k}}{(2\pi)^3} \left( e^{i\vec{k} \cdot \vec{r}} \frac{\omega^2\sin\omega t}{\omega} - \vec{k}^2 e^{i\vec{k} \cdot \vec{r}} \frac{\sin\omega t}{\omega} - m^2 e^{i\vec{k} \cdot \vec{r}} \frac{\sin\omega t}{\omega} \right) \\
        =& \int \frac{d^3\vec{k}}{(2\pi)^3} (\omega^2-\vec{k}^2-m^2) e^{i\vec{k} \cdot \vec{r}} \frac{\sin\omega t}{\omega}
    \end{align*}
    由$\omega = \sqrt{\vec{k}^2 + m^2}$可知:
    \[
        \left( -\frac{\partial^2}{\partial t^2} + \nabla^2 - m^2 \right)D(x) = 0
    \]
    下面验证初始条件,当$t=0$时,容易发现:
    \[
        D(x) = 0.
    \]
    计算对时间的一阶导数可得:
    \[
        \dot{D}(x) = \int \frac{d^3\vec{k}}{(2\pi)^3} e^{i\vec{k} \cdot \vec{r}} = \delta^3(\vec{r}).
    \]

\end{sol}


\end{document}