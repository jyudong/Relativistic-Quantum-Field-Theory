% !TEX program = pdflatex
% !TEX options = -synctex=1 -interaction=nonstopmode -file-line-error "%DOC%"
% 作业模板
\documentclass[UTF8,10pt,a4paper]{article}
\usepackage[UTF8]{ctex}% 中文支持
\usepackage{braket}% Q.M. 
\usepackage{enumerate}
\usepackage{slashed}
% 将全角句号映射为全角句点,需xelatex编译
% \catcode`\。=\active
% \newcommand{。}{.}
% 作业信息
\newcommand{\CourseName}{相对论量子场论}
\newcommand{\CourseCode}{PHYS2124}
\newcommand{\Semester}{22-23学年第学期}
\newcommand{\ProjectName}{作业四}
\newcommand{\DueTimeType}{截止日期}
\newcommand{\DueTime}{22.10.11}
\newcommand{\StudentName}{董建宇}
\newcommand{\StudentID}{2019511017}
\usepackage[vmargin=1in,hmargin=.5in]{geometry}% 页边距
\usepackage{fancyhdr}% 页眉和页脚
\usepackage{lastpage}% 返回末页页码
\usepackage{calc}% 返回文本宽度
\pagestyle{fancy}% 全局页眉和页脚风格
\fancyhf{}% 清除预设的页眉和页脚
\fancyhead[L]{\CourseName}% 页眉左
\fancyhead[C]{\ProjectName}% 页眉中
\fancyhead[R]{\StudentName}% 页眉右
\fancyfoot[R]{\thepage\ / \pageref{LastPage}}% 页脚右
\setlength\headheight{12pt}% 页眉高
\fancypagestyle{FirstPageStyle}{% 首页页眉和页脚风格
    \fancyhf{}% 清除预设的页眉和页脚
    \fancyhead[L]{\CourseName\\
        \CourseCode\\
        \Semester}% 页眉左
    \fancyhead[C]{{\Huge\bfseries\ProjectName}\\
        \DueTimeType\ : \DueTime}% 页眉中
    \fancyhead[R]{姓名 : \makebox[\widthof{\StudentID}][s]{\StudentName}\\
        学号 : \StudentID\\
        成绩 : \underline{\makebox[\widthof{\StudentID}]{}}}% 页眉右
    \fancyfoot[R]{\thepage\ / \pageref{LastPage}}% 页脚右
    \setlength\headheight{36pt}% 页眉高
}
\usepackage{amsmath,amssymb,amsthm,bm}% 基础数学支持,特殊数学字符,自定义定理,公式内加粗
\allowdisplaybreaks[4]% 公式跨页
\newtheoremstyle{Problem}% 定理风格名称
{}% 定理上方空间尺寸,留空为默认
{}% 定理下方空间尺寸,留空为默认
{}% 定理主体字体
{}% 定理缩进量
{\bfseries}% 定理开头字体
{.}% 定理开头后所接标点
{ }% 定理开头后所接空间尺寸,空格为默认词间距
{第\thmnumber{ #2}\thmname{ #1}\thmnote{ (#3)} 得分: \underline{\qquad\qquad}}% 定理开头格式,留空为默认
\theoremstyle{Problem}% 设定定理风格
\newtheorem{prob}{题}% 题目
\newtheoremstyle{Solution}% 定理风格名称
{}% 定理上方空间尺寸,留空为默认
{}% 定理下方空间尺寸,留空为默认
{}% 定理主体文本字体
{}% 定理缩进量
{\bfseries}% 定理开头字体
{:}% 定理开头后所接标点
{ }% 定理开头后所接空间尺寸,空格为默认词间距
{\thmname{#1}}% 定理开头格式
\makeatletter
\def\@endtheorem{\qed\endtrivlist\@endpefalse}% 题解后添加qed符号(黑色空心小正方形)
\makeatother
\theoremstyle{Solution}% 设定定理风格
\newtheorem*{sol}{解}% 题解
% \usepackage{mathrsfs}% 公式内花体字母 - \mathscr{}
% \usepackage{esint}% 特殊积分号
% \providecommand{\abs}[1]{\left\lvert#1\right\rvert}% 绝对值 - \abs{}
% \providecommand{\norm}[1]{\left\lVert#1\right\rVert}% 范数 - \norm{}
% \providecommand{\bra}[1]{\left\langle#1\right\rvert}% 左矢 - \bra{}
% \providecommand{\ket}[1]{\left\lvert#1\right\rangle}% 右矢 - \ket{}
% \providecommand{\braket}[2]{\left\langle#1\vert#2\right\rangle}% 右矢接左矢 - \braket{}{}
% \usepackage{graphicx}% 图片 -
% \begin{figure}[htbp]% 图片位置优先顺序:当地,页顶,页底,另起一页
%     \centering% 图片居中
%     \includegraphics[scale=.5]{图片文件名/路径名}
%     \caption{图片文字说明}
%     \label{图片引用代码}
% \end{figure}
% \usepackage{float}% 强制设定图片位置 - [H]
% \usepackage{subfigure}% figure环境内多子图 -
% \begin{figure}[htbp]% figure环境位置优先顺序:当地、页顶、页底、另起一页
%     \centering% figure环境居中
%     \subfigure[子图文字说明]{
%         \label{子图引用代码}
%         \includegraphics[width=0.45\textwidth]{子图文件名/路径}}
%     \subfigure[子图文字说明]{
%         \label{子图引用代码}
%         \includegraphics[width=0.45\textwidth]{子图文件名/路径}}
%     \caption{总文字说明}
%     \label{总引用代码}
% \end{figure}
% \usepackage{multirow}% 表格内多行单元格合并
% \usepackage{booktabs}% 三线表 - \toprule, \midrule, \bottomrule
% \usepackage{longtable}% 表格跨页 -
% \begin{center}% 表格居中
%     \begin{longtable}{lcr}% 表格列对齐: l = 居左, c = 居中, r = 居右
%         \caption{表格文字说明}
%         \label{表格引用代码}
%     \end{longtable}
% \end{center}
% \usepackage[version=4]{mhchem}% 化学式 - \ce{}


\begin{document}
\thispagestyle{FirstPageStyle}% 设定首页页眉和页脚风格
\begin{prob}对于自由Dirac场,证明下列非等时反对易关系
    \[
        \{ \psi_\alpha(x), \bar{\psi}_\beta(0) \} = i\left( \gamma_\mu\frac{\partial }{\partial x_\mu} - m \right)_{\alpha\beta} D(x)
    \]
    其中
    \[
        D(x) = \int \frac{d^3\vec{k}}{(2\pi)^3}e^{i\vec{k}\cdot\vec{x}} \frac{\sin\omega t}{\omega}
    \]
    \[
        \omega = \sqrt{\vec{k}^2+m^2}
    \]
\end{prob}

\begin{sol}
Dirac场算符为:
\[
    \psi_\alpha(\vec{r},t) = \frac{1}{\sqrt{\Omega}} \sum_{\vec{p},s} \left[ a_{\vec{p},s}(t) \left( u_{\vec{p},s} \right)_\alpha e^{i\vec{p} \cdot \vec{r}} + b^{\dagger}_{\vec{p},s}(t) \left( v_{\vec{p},s} \right)_\alpha e^{-i\vec{p}\cdot\vec{r}} \right].
\]
对于自由Dirac场,由Heisenberg运动方程可知:
\[
    \begin{aligned}
    a_{\vec{p},s}(t) &= a_{\vec{p},s}e^{-iE_pt} = a_{\vec{p},s} e^{-i\omega t}, ~~ a^\dagger_{\vec{p},s}(t) = a^\dagger_{\vec{p},s}e^{iE_pt} = a^\dagger_{\vec{p},s} e^{i\omega t}; \\
    b_{\vec{p},s}(t) &= b_{\vec{p},s}e^{-iE_pt} = b_{\vec{p},s} e^{-i\omega t}, ~~ a^\dagger_{\vec{p},s}(t) = b^\dagger_{\vec{p},s}e^{iE_pt} = b^\dagger_{\vec{p},s} e^{i\omega t}.
    \end{aligned}
\]
从而场算符为:
\[
    \begin{aligned}
    &\psi_\alpha(\vec{r},t) = \frac{1}{\sqrt{\Omega}} \sum_{\vec{p},s} \left[ a_{\vec{p},s} \left(u_{\vec{p},s}\right)_\alpha e^{i(\vec{p}\cdot\vec{r} - \omega t)} + b^\dagger_{\vec{p},s} \left(v_{\vec{p},s}\right)_\alpha e^{-i(\vec{p}\cdot\vec{r}-\omega t)} \right]; \\
    &\bar{\psi}_\beta(0,0) = \frac{1}{\sqrt{\Omega}} \sum_{\vec{p},s} \left[ a^\dagger_{\vec{p},s} \left(u_{\vec{p},s}^\dagger \gamma_4\right)_\beta + b_{\vec{p},s} \left(v_{\vec{p},s}^\dagger\gamma_4\right)_\beta \right]
    \end{aligned}
\]
可以计算非等时反对易关系如下:
\[
    \begin{aligned}
    \left\{ \psi_\alpha(x), \bar{\psi}_\beta(0) \right\} &= \frac{1}{\Omega} \sum_{\vec{p},s;\vec{p}',s'} \left\{ a_{\vec{p},s} \left(u_{\vec{p},s}\right)_\alpha e^{i(\vec{p}\cdot\vec{r} - \omega t)} + b^\dagger_{\vec{p},s} \left(v_{\vec{p},s}\right)_\alpha e^{-i(\vec{p}\cdot\vec{r}-\omega t)}, ~~ a^\dagger_{\vec{p}',s'} \left(u_{\vec{p}',s'}^\dagger \gamma_4\right)_\beta + b_{\vec{p},s} \left(v_{\vec{p}',s'}^\dagger \gamma_4\right)_\beta \right\} \\
    &= \frac{1}{\Omega} \sum_{\vec{p},s} \left[ \left( u_{\vec{p},s} \right)_\alpha \left(u_{\vec{p},s}^\dagger \gamma_4 \right)_\beta e^{-i\omega t} + \left( v_{-\vec{p},s} \right)_\alpha \left( v_{-\vec{p},s}^\dagger\gamma_4 \right)_\beta e^{i\omega t} \right] e^{i\vec{p}\cdot\vec{r}} \\
    &= \frac{1}{\Omega} \sum_{\vec{p}} \left[ \left(\sum_s u_{\vec{p},s} u_{\vec{p},s}^\dagger \beta\right)_{\alpha\beta}e^{-i\omega t} + \left(\sum_s v_{-\vec{p},s} v_{-\vec{p},s}^\dagger \beta\right)_{\alpha\beta}e^{i\omega t} \right] e^{i\vec{k}\cdot\vec{r}} \\
    &= \frac{1}{\Omega} \sum_{\vec{p}} \left[ \left(\frac{\slashed{p} + m}{2p_0}\right)_{\alpha\beta} e^{-i\omega t} + \left(\frac{-\slashed{p} - m}{2p_0}\right)_{\alpha\beta} e^{i\omega t} \right] e^{i\vec{k}\cdot\vec{r}} \\
    &= \frac{1}{\Omega} \sum_{\vec{p}} \left[ \left( \frac{-\slashed{p} - m}{\omega}\right)_{\alpha\beta}i\sin\omega t \right] e^{i\vec{k}\cdot\vec{r}} \\
    &= \frac{1}{\Omega} \sum_{\vec{p}} i\left( \gamma_\mu \frac{\partial }{\partial x_{\mu}} - m \right)_{\alpha\beta} \frac{\sin\omega t}{\omega} e^{i\vec{k}\cdot\vec{r}} \\
    &= i\left( \gamma_\mu \frac{\partial }{\partial x_{\mu}} - m \right)_{\alpha\beta} \int \frac{d^3\vec{k}}{(2\pi)^3} e^{i\vec{k}\cdot\vec{x}} \frac{\sin\omega t}{\omega}.
    \end{aligned}
\]
其中,利用了$p_0 = \sqrt{\vec{k}^2 + m^2} = \omega$和$\slashed{p} = -i\gamma_mu p_\mu = -\gamma_\mu \frac{\partial}{\partial x_\mu}$。 \\
即有:
\[
    \left\{ \psi_\alpha(x), \bar{\psi}_\beta(0) \right\} = i\left( \gamma_\mu \frac{\partial }{\partial x_{\mu}} - m \right)_{\alpha\beta} D(x).
\]
% 当$\alpha \neq \beta$时,容易得到$\left\{ \psi_\alpha(x), \bar{\psi}_\beta(0) \right\} = 0$。 \\
% 当$\alpha = \beta = 1$时,有:
% \[
%     \left\{ \psi_\alpha(x), \bar{\psi}_\beta(0) \right\} = \frac{1}{\Omega} \sum_{\vec{p}} \left( \cos\omega t - i\frac{m}{\omega}\sin\omega t \right) e^{i\vec{k}\cdot\vec{x}}
% \]
% 当$\alpha = \beta = 2$时,有:
% \[
%     \left\{ \psi_\alpha(x), \bar{\psi}_\beta(0) \right\} = \frac{1}{\Omega} \sum_{\vec{p}} \left( -\cos\omega t - i\frac{m}{\omega}\sin\omega t \right) e^{i\vec{k}\cdot\vec{x}}
% \]
\end{sol}


\begin{prob}证明$\bar{\psi}(x)\gamma_\mu\psi(x)$为一Lorentz矢量。

\end{prob}

\begin{sol}
进行坐标变换之后满足:
\[
    \psi'(x') = D(a) \psi(x);
\]
可以计算得:
\[
    \bar{\psi}'(x') = \psi'^\dagger \gamma_4 = \psi^\dagger(x) D^\dagger(a)\gamma_4 = \psi^\dagger(x) \gamma_4 D^{-1}(a).
\]
则有:
\[
    \bar{\psi}'(x') \gamma_\mu \psi'(x') = \psi^\dagger(x) \gamma_4 D^{-1}(a)\gamma_\mu D(a) \psi(x) = \psi^\dagger(x) \gamma_4 a_{\mu\nu}\gamma_\nu \psi(x) = a_{\mu\nu} \psi^\dagger(x) \gamma_4 \gamma_\nu \psi(x).
\]
即$\bar{\psi}(x)\gamma_\mu\psi(x)$为一Lorentz矢量。
\end{sol}


\begin{prob}证明电荷共轭旋量$\psi^c(x) = \gamma_2\psi^*(x)$与$\psi(x)$的Lorentz变换相同。


\end{prob}

\begin{sol}
要证明$\psi^c(x) = \gamma_2\psi^*(x)$与$\psi(x)$的Lorentz变换相同,即需要证明存在$D(a)$满足:
\[
    \psi'(x') = D(a)\psi(x); ~~ \psi'^c(x') = D(a)\psi^c(x).
\]
第一个方程两边取复共轭,有:
\[
    \psi'{}^* (x') = D^*(a)\psi^*(x).
\]
带入第二个方程得:
\[
    \psi'^c(x') = \gamma_2\psi'^*(x') = \gamma_2 D^*(a)\psi^*(x) = D(a)\gamma_2\psi^*(x).
\]
即只需要证明:
\[
    \gamma_2D^*(a) = D(a)\gamma_2.
\]
考虑无穷小Lorentz变换的$D(a) = 1+\frac{i}{4}\varepsilon_{\mu\nu}\sigma_{\mu\nu}$。其中$\varepsilon_{ij} = -\varepsilon_{ji}$为实数;$\varepsilon_{j4} = -\varepsilon_{4j}$为虚数;$\varepsilon_{\mu\mu} = 0$。则可以得到$D(a)$为:
\[
    D(a) = 1 + \frac{i}{4}(\varepsilon_{ij}\sigma_{ij} + \varepsilon_{j4}\sigma_{j4} + \varepsilon_{4j}\sigma_{4j}) = 1 + \frac{i}{2}\left( \sum_{i<j} \varepsilon_{ij}\sigma_{ij} + \sum_{j=1}^3 \varepsilon_{j4}\sigma_{j4} \right).
\]
由Pauli矩阵的表达式可以计算:
\[
    \gamma_1^* = -\gamma_1; ~~ \gamma_2^* = \gamma_2; ~~ \gamma_3^* = -\gamma_3; ~~ \gamma_4^* = \gamma_4.
\]
进而可以计算:
\[
    \sigma_{12}^* = -\frac{1}{2i}(\gamma_1^* \gamma_2^* - \gamma_2^*\gamma_1^*) = \sigma_{12}; ~~ \sigma_{13}^* = -\sigma_{13}; ~~ \sigma_{23}^* = \sigma_{23}; ~~ \sigma_{14}^* = \sigma_{14}; ~~ \sigma_{24}^* = -\sigma_{24}; ~~ \sigma_{34}^* = \sigma_{34}
\]
则有:
\[
    D^*(a) = 1 - \frac{i}{2}\left( \varepsilon_{12}\sigma_{12} - \varepsilon_{13}\sigma_{13} + \varepsilon_{23}\sigma_{23} - \varepsilon_{14}\sigma_{14} + \varepsilon_{24}\sigma_{24} - \varepsilon_{34}\sigma_{34} \right).
\]
可以计算得:
\[
    \begin{aligned}
    &\gamma_2 \sigma_{12} = \frac{1}{2i}(\gamma_2\gamma_1\gamma_2 - \gamma_2^2\gamma_1) = \frac{1}{2i}(\gamma_2\gamma_1 - \gamma_1\gamma_2)\gamma_2 = -\sigma_{12}\gamma_2; \\
    &\gamma_2 \sigma_{13} = \frac{1}{2i}(\gamma_2\gamma_1\gamma_3 - \gamma_2\gamma_3\gamma_1) = \frac{1}{2i}(\gamma_1\gamma_3-\gamma_3\gamma_1)\gamma_2 = \sigma_{13}\gamma_2; \\
    &\gamma_2\sigma_{23} = \frac{1}{2i}(\gamma_2^2\gamma_3-\gamma_2\gamma_3\gamma_2) = \frac{1}{2i}(\gamma_3\gamma_2 - \gamma_2\gamma_3)\gamma_2 = -\sigma_{23}\gamma_2; \\
    &\gamma_2\sigma_{14} = \frac{1}{2i}(\gamma_2\gamma_1\gamma_4 - \gamma_4\gamma_1\gamma_2) = \frac{1}{2i}(\gamma_1\gamma_4-\gamma_4\gamma_1)\gamma_2 = \sigma_{14}\gamma_2; \\
    &\gamma_2 \sigma_{24} = \frac{1}{2i}(\gamma_2^2\gamma_4 - \gamma_2\gamma_4\gamma_2) = \frac{1}{2i}(\gamma_4\gamma_2-\gamma_2\gamma_4)\gamma_2 = -\sigma_{24}\gamma_2; \\
    &\gamma_2\sigma_{34} = \frac{1}{2i}(\gamma_2\gamma_3\gamma_4 - \gamma_2\gamma_4\gamma_3) = \frac{1}{2i}(\gamma_3\gamma_4-\gamma_4\gamma_3)\gamma_2 = \sigma_{34}\gamma_2.
    \end{aligned}
\]
从而有:
\[
    \gamma_2 D^*(a) = \gamma_2 + \frac{i}{2}(\varepsilon_{12}\sigma_{12} + \varepsilon_{13}\sigma_{13} + \varepsilon_{23}\sigma_{23} + \varepsilon_{14}\sigma_{14} + \varepsilon_{24}\sigma_{24} + \varepsilon_{34}\sigma_{34})\gamma_2 = D(a)\gamma_2.
\]
综上所述,电荷共轭旋量$\psi^c(x) = \gamma_2\psi^*(x)$与$\psi(x)$的Lor
\end{sol}

\end{document}