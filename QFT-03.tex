% !TEX program = pdflatex
% !TEX options = -synctex=1 -interaction=nonstopmode -file-line-error "%DOC%"
% 作业模板
\documentclass[UTF8,10pt,a4paper]{article}
\usepackage[UTF8]{ctex}% 中文支持
\usepackage{braket}% Q.M. 
\usepackage{enumerate}
\usepackage{slashed} %字母画斜线 \slashed{D}
% 将全角句号映射为全角句点,需xelatex编译
% \catcode`\。=\active
% \newcommand{。}{.}
% 作业信息
\newcommand{\CourseName}{相对论量子场论}
\newcommand{\CourseCode}{PHYS2124}
\newcommand{\Semester}{22-23学年秋学期}
\newcommand{\ProjectName}{作业三}
\newcommand{\DueTimeType}{截止日期}
\newcommand{\DueTime}{22.10.04}
\newcommand{\StudentName}{董建宇}
\newcommand{\StudentID}{2019511017}
\usepackage[vmargin=1in,hmargin=.5in]{geometry}% 页边距
\usepackage{fancyhdr}% 页眉和页脚
\usepackage{lastpage}% 返回末页页码
\usepackage{calc}% 返回文本宽度
\pagestyle{fancy}% 全局页眉和页脚风格
\fancyhf{}% 清除预设的页眉和页脚
\fancyhead[L]{\CourseName}% 页眉左
\fancyhead[C]{\ProjectName}% 页眉中
\fancyhead[R]{\StudentName}% 页眉右
\fancyfoot[R]{\thepage\ / \pageref{LastPage}}% 页脚右
\setlength\headheight{12pt}% 页眉高
\fancypagestyle{FirstPageStyle}{% 首页页眉和页脚风格
    \fancyhf{}% 清除预设的页眉和页脚
    \fancyhead[L]{\CourseName\\
        \CourseCode\\
        \Semester}% 页眉左
    \fancyhead[C]{{\Huge\bfseries\ProjectName}\\
        \DueTimeType\ : \DueTime}% 页眉中
    \fancyhead[R]{姓名 : \makebox[\widthof{\StudentID}][s]{\StudentName}\\
        学号 : \StudentID\\
        成绩 : \underline{\makebox[\widthof{\StudentID}]{}}}% 页眉右
    \fancyfoot[R]{\thepage\ / \pageref{LastPage}}% 页脚右
    \setlength\headheight{36pt}% 页眉高
}
\usepackage{amsmath,amssymb,amsthm,bm}% 基础数学支持,特殊数学字符,自定义定理,公式内加粗
\allowdisplaybreaks[4]% 公式跨页
\newtheoremstyle{Problem}% 定理风格名称
{}% 定理上方空间尺寸,留空为默认
{}% 定理下方空间尺寸,留空为默认
{}% 定理主体字体
{}% 定理缩进量
{\bfseries}% 定理开头字体
{.}% 定理开头后所接标点
{ }% 定理开头后所接空间尺寸,空格为默认词间距
{第\thmnumber{ #2}\thmname{ #1}\thmnote{ (#3)} 得分: \underline{\qquad\qquad}}% 定理开头格式,留空为默认
\theoremstyle{Problem}% 设定定理风格
\newtheorem{prob}{题}% 题目
\newtheoremstyle{Solution}% 定理风格名称
{}% 定理上方空间尺寸,留空为默认
{}% 定理下方空间尺寸,留空为默认
{}% 定理主体文本字体
{}% 定理缩进量
{\bfseries}% 定理开头字体
{:}% 定理开头后所接标点
{ }% 定理开头后所接空间尺寸,空格为默认词间距
{\thmname{#1}}% 定理开头格式
\makeatletter
\def\@endtheorem{\qed\endtrivlist\@endpefalse}% 题解后添加qed符号(黑色空心小正方形)
\makeatother
\theoremstyle{Solution}% 设定定理风格
\newtheorem*{sol}{解}% 题解
% \usepackage{mathrsfs}% 公式内花体字母 - \mathscr{}
% \usepackage{esint}% 特殊积分号
% \providecommand{\abs}[1]{\left\lvert#1\right\rvert}% 绝对值 - \abs{}
% \providecommand{\norm}[1]{\left\lVert#1\right\rVert}% 范数 - \norm{}
% \providecommand{\bra}[1]{\left\langle#1\right\rvert}% 左矢 - \bra{}
% \providecommand{\ket}[1]{\left\lvert#1\right\rangle}% 右矢 - \ket{}
% \providecommand{\braket}[2]{\left\langle#1\vert#2\right\rangle}% 右矢接左矢 - \braket{}{}
% \usepackage{graphicx}% 图片 -
% \begin{figure}[htbp]% 图片位置优先顺序:当地,页顶,页底,另起一页
%     \centering% 图片居中
%     \includegraphics[scale=.5]{图片文件名/路径名}
%     \caption{图片文字说明}
%     \label{图片引用代码}
% \end{figure}
% \usepackage{float}% 强制设定图片位置 - [H]
% \usepackage{subfigure}% figure环境内多子图 -
% \begin{figure}[htbp]% figure环境位置优先顺序:当地、页顶、页底、另起一页
%     \centering% figure环境居中
%     \subfigure[子图文字说明]{
%         \label{子图引用代码}
%         \includegraphics[width=0.45\textwidth]{子图文件名/路径}}
%     \subfigure[子图文字说明]{
%         \label{子图引用代码}
%         \includegraphics[width=0.45\textwidth]{子图文件名/路径}}
%     \caption{总文字说明}
%     \label{总引用代码}
% \end{figure}
% \usepackage{multirow}% 表格内多行单元格合并
% \usepackage{booktabs}% 三线表 - \toprule, \midrule, \bottomrule
% \usepackage{longtable}% 表格跨页 -
% \begin{center}% 表格居中
%     \begin{longtable}{lcr}% 表格列对齐: l = 居左, c = 居中, r = 居右
%         \caption{表格文字说明}
%         \label{表格引用代码}
%     \end{longtable}
% \end{center}
% \usepackage[version=4]{mhchem}% 化学式 - \ce{}


\begin{document}
\thispagestyle{FirstPageStyle}% 设定首页页眉和页脚风格
\begin{prob}
    \begin{enumerate}[(i)]
        \item 证明下列旋量波函数的表达式:
        \[
            u_{\vec{p},s} = \eta \left(\begin{array}{@{}c}
                \sqrt{\frac{E_p+m}{2E_p}}\varphi_s \\
                2s\sqrt{\frac{E_p-m}{2E_p}}\varphi_s
            \end{array}
            \right) 
            ~~ 
            v_{-\vec{p},s} = \eta' \left(\begin{array}{@{}c}
                -2s\sqrt{\frac{E_p-m}{2E_p}} \varphi_s \\
                \sqrt{\frac{E_p+m}{2E_p}}\varphi_s
            \end{array}\right)
        \]
        其中$E_p = \sqrt{\vec{p}^2 + m^2}$,$\eta,\eta'$为任意相因子,$\vert \eta \vert = \vert \eta' \vert = 1$,$2\times 1$列矩阵$\varphi_s$为$\vec{\tau}\cdot\hat{p}$的本征矢量,即
        \[
            \vec{\tau}\cdot \hat{p} \varphi_s = 2s \varphi_s
        \]

        \item 讨论上述旋量波函数的非相对论极限。
    \end{enumerate}
\end{prob}

\begin{sol}
(i)旋量波函数$u_{\vec{p},s}$满足
\[
    \left\{ \begin{aligned}
    &(\vec{\alpha} \cdot \vec{p} + \beta m) u_{\vec{p},s} = E_p u_{\vec{p},s}, \\
    &\vec{\sigma} \cdot \hat{p} u_{\vec{p},s} = 2s u_{\vec{p},s}.
    \end{aligned} \right.
\]
其中$\vec{\alpha} = \rho_1 \vec{\sigma} = \begin{pmatrix}
    0 & \vec{\tau} \\
    \vec{\tau} & 0
\end{pmatrix}, \beta = \begin{pmatrix}
    I & 0 \\
    0 & -I
\end{pmatrix}$。令$u_{\vec{p},s} = \begin{pmatrix}
    u_1 \\
    u_2
\end{pmatrix}$,其中$u_1,u_2$分别表示两个$2\times 1$列矩阵。则上述方程可以写为:
\[
    \begin{pmatrix}
        m & \vec{\tau} \cdot \vec{p} \\
        \vec{\tau} \cdot \vec{p} & -m
    \end{pmatrix}
    \begin{pmatrix}
        u_1 \\
        u_2
    \end{pmatrix} = 
    E_p \begin{pmatrix}
        u_1 \\
        u_2
    \end{pmatrix}; 
    ~~ 
    \begin{pmatrix}
        \vec{\tau}\cdot\hat{p} & 0 \\
        0 & \vec{\tau}\cdot\hat{p}
    \end{pmatrix}
    \begin{pmatrix}
        u_1 \\
        u_2
    \end{pmatrix} = 
    2s\begin{pmatrix}
        u_1 \\
        u_2
    \end{pmatrix}.
\]
可以验证:
\[
    \begin{aligned}
    &\eta \left( m\sqrt{\frac{E_p+m}{2E_p}} + (2s)^2 \vert \vec{p} \vert \sqrt{\frac{E_p-m}{2E_p}} \right)\varphi_s = \eta \sqrt{\frac{E_p+m}{2E_p}} \left( m + \vert \vec{p} \vert \frac{E_p-m}{\sqrt{E_p^2 - m^2}} \right) \varphi_s = \eta \sqrt{\frac{E_p+m}{2E_p}}\varphi_s; \\
    &\eta \left( 2s \vert \vec{p} \vert \sqrt{\frac{E_p+m}{2E_p}} -m 2s\sqrt{\frac{E_p-m}{2E_p}} \right) \varphi_s = \eta(2s) \sqrt{\frac{E_p-m}{2E_p}} \left( \vert \vec{p} \vert \frac{E_p+m}{\sqrt{E_p^2-m^2}} - m \right)\varphi_s = \eta 2s \sqrt{\frac{E_p-m}{2E_p}} \varphi_s.
    \end{aligned}
\]
其中,利用了$\sqrt{E_p^2 - m^2} = \vert \vec{p} \vert$和$2s = \pm 1,4s^2 = 1$。 \\
同理,对于$v_{-\vec{p},s}$,Dirac Hamiltonian为:
\[
    \left\{ \begin{aligned}
    &(\vec{\alpha} \cdot \vec{p} + \beta m) v_{-\vec{p},s} = -E_p u_{-\vec{p},s}, \\
    &\vec{\sigma} \cdot \hat{p} v_{-\vec{p},s} = 2s v_{-\vec{p},s}.
    \end{aligned} \right.
\]
可以验证:
\[
    \begin{aligned}
    &\eta' \left( -2sm\sqrt{\frac{E_p-m}{2E_p}} + 2s \vert \vec{p} \vert \sqrt{\frac{E_p+m}{2E_p}} \right)\varphi_s = \eta' (-2s) \sqrt{\frac{E_p-m}{2E_p}} \left( m - \vert \vec{p} \vert \frac{E_p+m}{\sqrt{E_p^2 - m^2}} \right)\varphi_s = -E_p \eta' (-2s) \sqrt{\frac{E_p-m}{2E_p}} \varphi_s; \\
    &\eta' \left( -(2s)^2\vert \vec{p} \vert \sqrt{\frac{E_p - m}{2E_p}} - m\sqrt{\frac{E_p+m}{2E_p}} \right)\varphi_s = -\eta' \sqrt{\frac{E_p+m}{2E_p}} \left( \vert \vec{p} \vert \frac{E_p-m}{\sqrt{E_p^2-m^2}} + m \right) \varphi_s = -\eta' \sqrt{\frac{E_p+m}{2E_p}} \varphi_s.
    \end{aligned}
\]
(ii)对于非相对论极限,有$m >> \vert \vec{p} \vert$,即有:
\[
    \frac{E_p}{m} = \sqrt{1 + \frac{\vec{p}^2}{m^2}} = 1 + \frac{\vec{p}^2}{2m^2} + \mathcal{O}(\vec{p}^2/m^2).
\]
则有:
\[
    \left\{\begin{aligned}
    &\frac{E_p+m}{2E_p} = \frac{1}{2} + \frac{1}{2(E_p/m)} = \frac{1}{2} + \frac{1}{2}\left( 1 - \frac{\vec{p}^2}{2m^2} \right) = 1 - \frac{\vec{p}^2}{4m^2}; \\
    &\frac{E_p-m}{2E_p} = \frac{1}{2} - \frac{1}{2(E_p/m)} = \frac{1}{2} - \frac{1}{2}\left( 1 - \frac{\vec{p}^2}{2m^2} \right) = \frac{\vec{p}^2}{4m^2}.
    \end{aligned}\right.
\]
进而有:
\[
    \left\{\begin{aligned}
    &\sqrt{\frac{E_p+m}{2E_p}} = \sqrt{1 - \frac{\vec{p}^2}{4m^2}} = 1 - \frac{\vec{p}^2}{8m^2}; \\
    &\sqrt{\frac{E_p-m}{2E_p}} = \frac{\vert \vec{p} \vert}{2m}.
    \end{aligned}\right.
\]
则上述旋量波函数的非相对论极限为:
\[
    u_{\vec{p},s} = \eta \begin{pmatrix}
        \sqrt{1 - \frac{\vec{p}^2}{4m^2}} \varphi_s \\
        s\frac{\vert \vec{p} \vert}{m}\varphi_s
    \end{pmatrix}
    ~~
    v_{-\vec{p},s} = \eta' \begin{pmatrix}
        -s\frac{\vert \vec{p} \vert}{m} \varphi_s \\
        \sqrt{1 - \frac{\vec{p}^2}{4m^2}} \varphi_s
    \end{pmatrix}
\]
\end{sol}

\begin{prob}[]证明
    \begin{align*}
        \sum_s u_{\vec{p},s} u_{\vec{p},s}^\dagger \beta = \frac{\slashed{p}+m}{2p_0} \\
        \sum_s v_{\vec{p},s} v_{\vec{p},s}^\dagger \beta = \frac{\slashed{p}-m}{2p_0}
    \end{align*}
    其中$\slashed{p} = -i\gamma_\mu p_\mu$,$p_4 = ip_0 = i\sqrt{\vec{p}^2 + m^2}$。
\end{prob}

\begin{sol}
可以计算得:
\[
    u_{\vec{p},s} u_{\vec{p},s}^\dagger \beta = 
    \begin{pmatrix}
        \sqrt{\frac{E_p+m}{2E_p}} \varphi_s \\
        2s\sqrt{\frac{E_p-m}{2E_p}} \varphi_s
    \end{pmatrix}
    \begin{pmatrix}
        \sqrt{\frac{E_p+m}{2E_p}} \varphi_s^\dagger & 2s \sqrt{\frac{E_p-m}{2E_p}} \varphi_s^\dagger
    \end{pmatrix}
    \begin{pmatrix}
        I & 0 \\
        0 & -I
    \end{pmatrix} = 
    \begin{pmatrix}
        \frac{E_p+m}{2E_p}\varphi_s\varphi_s^\dagger & -s\frac{\vert \vec{p} \vert}{E_p} \varphi_s\varphi_s^\dagger \\
        s\frac{\vert \vec{p} \vert}{E_p}\varphi_s\varphi_s^\dagger & -\frac{E_p-m}{2E_p}\varphi_s\varphi_s^\dagger
    \end{pmatrix}
\]
对于等号右侧,可以计算:
\[
    \frac{\slashed{p} + m}{2p_0} = \frac{1}{2E_p} (-i\vec{\gamma} \cdot \vec{p} - i\gamma_4p_4 + m) = \frac{1}{2E_p} \begin{pmatrix}
        E_p + m & -\vec{\tau} \cdot \vec{p} \\
        \vec{\tau} \cdot \vec{p} & m-E_p
    \end{pmatrix}
\]
注意到:
\[
    \sum_s s\vert \vec{p} \vert \varphi_s \varphi_s^\dagger = \sum_s \tau \cdot \vec{p} \varphi_s \varphi_s^\dagger = \tau \cdot \vec{p} \sum_s \varphi_s \varphi_s^\dagger = \tau \cdot \vec{p}.
\]
其中,利用了$\sum_s \varphi_s \varphi_s^\dagger = I$。则有:
\[
    \sum_s u_{\vec{p},s} u_{\vec{p},s}^\dagger \beta = \frac{1}{2E_p} \begin{pmatrix}
        (E_p+m) \sum_s \varphi_s \varphi_s^\dagger & \sum_s -2s \vert \vec{p} \vert \varphi_s \varphi_s^\dagger \\
        \sum_s 2s\vert \vec{p} \vert \varphi_s \varphi_s^\dagger & (m-E_p) \sum_s \varphi_s \varphi_s^\dagger 
    \end{pmatrix} = \frac{1}{2E_p}
    \begin{pmatrix}
        E_p+m & -\tau \cdot \vec{p} \\
        \tau \cdot \vec{p} & m-E_p
    \end{pmatrix} = \frac{\slashed{p} + m}{2p_0}
\]
同理,对于第二个方程,可以计算:
\[
    v_{\vec{p},s}v_{\vec{p},s}^\dagger \beta = \begin{pmatrix}
        -2s\sqrt{\frac{E_p-m}{2E_p}}\varphi_s \\
        \sqrt{\frac{E_p+m}{2E_p}}\varphi_s
    \end{pmatrix}
    \begin{pmatrix}
        -2s\sqrt{\frac{E_p-m}{2E_p}}\varphi_s^\dagger & \sqrt{\frac{E_p+m}{2E_p}}\varphi_s^\dagger
    \end{pmatrix}
    \begin{pmatrix}
        I & 0 \\
        0 & -I
    \end{pmatrix} = 
    \begin{pmatrix}
        \frac{E_p - m}{2E_p} \varphi_s \varphi_s^\dagger & s \frac{\vert \vec{p} \vert}{E_p} \varphi_s \varphi_s^\dagger \\
        -s\frac{\vert \vec{p} \vert}{E_p} \varphi_s \varphi_s^\dagger & -\frac{E_p+m}{2E_p}\varphi_s \varphi_s^\dagger
    \end{pmatrix}
\]
求和得到:
\[
    \sum_s v_{\vec{p},s} v_{\vec{p},s}^\dagger \beta = \frac{1}{2E_p} \begin{pmatrix}
        (E_p-m)\sum_s \varphi_s \varphi_s^\dagger & \sum_s 2s \vert \vec{p} \vert \varphi_s \varphi_s^\dagger \\
        \sum_s -2s\vert \vec{p} \vert \varphi_s \varphi_s^\dagger & -(E_p+m) \sum_s \varphi_s \varphi_s^\dagger
    \end{pmatrix} = \frac{1}{2E_p} 
    \begin{pmatrix}
        E_p - m & \tau \cdot \vec{p} \\
        -\tau \cdot \vec{p} & -E_p-m
     \end{pmatrix}
\]
对于等号右侧,可以计算:
\[
    \frac{\slashed{p} - m}{2p_0} = \frac{1}{2E_p} (-i\vec{\gamma}\cdot\vec{p} - i\gamma_4p_4 - m) = \frac{1}{2E_p}\begin{pmatrix}
        E_p - m & \vec{\tau} \cdot \vec{p} \\
        -\vec{\tau} \cdot \vec{p} & - E_p - m
    \end{pmatrix} = \sum_s v_{\vec{p},s} v_{\vec{p},s}^\dagger \beta.
\]
\end{sol}

\end{document}