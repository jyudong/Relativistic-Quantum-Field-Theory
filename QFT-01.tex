% !TEX program = pdflatex
% !TEX options = -synctex=1 -interaction=nonstopmode -file-line-error "%DOC%"
% 作业模板
\documentclass[UTF8,10pt,a4paper]{article}
\usepackage[UTF8]{ctex}% 中文支持
\usepackage{braket}% Q.M. 
\usepackage{enumerate}
% 将全角句号映射为全角句点,需xelatex编译
% \catcode`\。=\active
% \newcommand{。}{.}
% 作业信息
\newcommand{\CourseName}{相对论量子场论}
\newcommand{\CourseCode}{PHYS2124}
\newcommand{\Semester}{22-23学年秋学期}
\newcommand{\ProjectName}{作业一}
\newcommand{\DueTimeType}{截止日期}
\newcommand{\DueTime}{22.09.20}
\newcommand{\StudentName}{董建宇}
\newcommand{\StudentID}{2019511017}
\usepackage[vmargin=1in,hmargin=.5in]{geometry}% 页边距
\usepackage{fancyhdr}% 页眉和页脚
\usepackage{lastpage}% 返回末页页码
\usepackage{calc}% 返回文本宽度
\pagestyle{fancy}% 全局页眉和页脚风格
\fancyhf{}% 清除预设的页眉和页脚
\fancyhead[L]{\CourseName}% 页眉左
\fancyhead[C]{\ProjectName}% 页眉中
\fancyhead[R]{\StudentName}% 页眉右
\fancyfoot[R]{\thepage\ / \pageref{LastPage}}% 页脚右
\setlength\headheight{12pt}% 页眉高
\fancypagestyle{FirstPageStyle}{% 首页页眉和页脚风格
    \fancyhf{}% 清除预设的页眉和页脚
    \fancyhead[L]{\CourseName\\
        \CourseCode\\
        \Semester}% 页眉左
    \fancyhead[C]{{\Huge\bfseries\ProjectName}\\
        \DueTimeType\ : \DueTime}% 页眉中
    \fancyhead[R]{姓名 : \makebox[\widthof{\StudentID}][s]{\StudentName}\\
        学号 : \StudentID\\
        成绩 : \underline{\makebox[\widthof{\StudentID}]{}}}% 页眉右
    \fancyfoot[R]{\thepage\ / \pageref{LastPage}}% 页脚右
    \setlength\headheight{36pt}% 页眉高
}
\usepackage{amsmath,amssymb,amsthm,bm}% 基础数学支持,特殊数学字符,自定义定理,公式内加粗
\allowdisplaybreaks[4]% 公式跨页
\newtheoremstyle{Problem}% 定理风格名称
{}% 定理上方空间尺寸,留空为默认
{}% 定理下方空间尺寸,留空为默认
{}% 定理主体字体
{}% 定理缩进量
{\bfseries}% 定理开头字体
{.}% 定理开头后所接标点
{ }% 定理开头后所接空间尺寸,空格为默认词间距
{第\thmnumber{ #2}\thmname{ #1}\thmnote{ (#3)} 得分: \underline{\qquad\qquad}}% 定理开头格式,留空为默认
\theoremstyle{Problem}% 设定定理风格
\newtheorem{prob}{题}% 题目
\newtheoremstyle{Solution}% 定理风格名称
{}% 定理上方空间尺寸,留空为默认
{}% 定理下方空间尺寸,留空为默认
{}% 定理主体文本字体
{}% 定理缩进量
{\bfseries}% 定理开头字体
{:}% 定理开头后所接标点
{ }% 定理开头后所接空间尺寸,空格为默认词间距
{\thmname{#1}}% 定理开头格式
\makeatletter
\def\@endtheorem{\qed\endtrivlist\@endpefalse}% 题解后添加qed符号(黑色空心小正方形)
\makeatother
\theoremstyle{Solution}% 设定定理风格
\newtheorem*{sol}{解}% 题解
% \usepackage{mathrsfs}% 公式内花体字母 - \mathscr{}
% \usepackage{esint}% 特殊积分号
% \providecommand{\abs}[1]{\left\lvert#1\right\rvert}% 绝对值 - \abs{}
% \providecommand{\norm}[1]{\left\lVert#1\right\rVert}% 范数 - \norm{}
% \providecommand{\bra}[1]{\left\langle#1\right\rvert}% 左矢 - \bra{}
% \providecommand{\ket}[1]{\left\lvert#1\right\rangle}% 右矢 - \ket{}
% \providecommand{\braket}[2]{\left\langle#1\vert#2\right\rangle}% 右矢接左矢 - \braket{}{}
% \usepackage{graphicx}% 图片 -
% \begin{figure}[htbp]% 图片位置优先顺序:当地,页顶,页底,另起一页
%     \centering% 图片居中
%     \includegraphics[scale=.5]{图片文件名/路径名}
%     \caption{图片文字说明}
%     \label{图片引用代码}
% \end{figure}
% \usepackage{float}% 强制设定图片位置 - [H]
% \usepackage{subfigure}% figure环境内多子图 -
% \begin{figure}[htbp]% figure环境位置优先顺序:当地、页顶、页底、另起一页
%     \centering% figure环境居中
%     \subfigure[子图文字说明]{
%         \label{子图引用代码}
%         \includegraphics[width=0.45\textwidth]{子图文件名/路径}}
%     \subfigure[子图文字说明]{
%         \label{子图引用代码}
%         \includegraphics[width=0.45\textwidth]{子图文件名/路径}}
%     \caption{总文字说明}
%     \label{总引用代码}
% \end{figure}
% \usepackage{multirow}% 表格内多行单元格合并
% \usepackage{booktabs}% 三线表 - \toprule, \midrule, \bottomrule
% \usepackage{longtable}% 表格跨页 -
% \begin{center}% 表格居中
%     \begin{longtable}{lcr}% 表格列对齐: l = 居左, c = 居中, r = 居右
%         \caption{表格文字说明}
%         \label{表格引用代码}
%     \end{longtable}
% \end{center}
% \usepackage[version=4]{mhchem}% 化学式 - \ce{}


\begin{document}
\thispagestyle{FirstPageStyle}% 设定首页页眉和页脚风格
\begin{prob}[最大-最小原理(maximum-minimum principle)]
    令H为一有下界的Hermitian算符,其本征值为$E_0 \leq E_1 \leq E_2 \leq \cdots$,对应的本征态为$\ket{0}, \ket{1}, \ket{2}, \cdots$。
    \vspace{-0.5em}
    \begin{enumerate}[i)]
        \item 令$\ket{b}$为一任意态矢量,$F(b)$为
        \[
            \frac{\bra{~~} H \ket{~~}}{\langle ~~ \vert ~~ \rangle}
        \]
        满足条件$\langle b \vert ~~ \rangle = 0$的最小值。改变$\ket{b}$证明$F(b)$的最大值为$E_1$。 \\
        提示:$F(b)$可以通过下列态矢量$\ket{~~} = \langle b \vert 1 \rangle \ket{0} - \langle b \vert 0 \rangle \ket{1}$得到。
        \item 令$\ket{b_1}, \ket{b_2}, \cdots \ket{b_n}$为任意态矢量,$F(b_1,b_2,\cdots,b_n)$为
        \[
            \frac{\bra{~~} H \ket{~~}}{\langle ~~ \vert ~~ \rangle}
        \]
        满足条件$\langle b_1 \vert ~~ \rangle = \langle b_2 \vert ~~ \rangle =  \cdots = \langle b_n \vert ~~ \rangle = 0$的最小值。证明$F(b_1,b_2,\cdots,b_n)$的最大值为$E_n$。
    \end{enumerate}
\end{prob}
\begin{sol}
\begin{enumerate}[i)]
    \item 对于态矢量$\ket{~~}$,总可以写成
    \[
        \ket{~~} = \sum_{i=0}^\infty c_i \ket{i}.
    \]
    可以计算$F(b)$为
    \[
        F(b) = \frac{\sum_{i=0}^{\infty} \vert c_i \vert^2 E_i}{\sum_{i=0}^{\infty} \vert c_i \vert^2}.
    \]
    要使得$F(b)$最小,则需要使$\ket{~~}$处在由尽可能少且能量尽可能低的本征态张成的子空间中。此时,可以注意到,当$F(b)$最小时,$\ket{~~} = \ket{0}$。但是由于$\ket{b}$是任意的,可以选取$\ket{b} = \ket{0}$使得$\langle b \vert ~~ \rangle = 0$的条件不成立。所以$\ket{~~}$不能在仅由$\ket{0}$张成的子空间中。其次,考虑$\ket{~~}$处在$\ket{0}$与$\ket{1}$张成的子空间中,可以选取
    \[
        \ket{~~} = \langle b \vert 1 \rangle \ket{0} - \langle b \vert 0 \rangle \ket{1}
    \]
    满足$\langle b \vert ~~ \rangle = 0$。则此时可以计算$F(b)$为:
    \[
        F(b) = \frac{E_0 \Vert \langle b \vert 1 \rangle \Vert^2 + E_1 \Vert \langle b \vert 0 \rangle \Vert^2}{\Vert \langle b \vert 1 \rangle \Vert^2 + \Vert \langle b \vert 0 \rangle \Vert^2} \leq \frac{E_1 \left( \Vert \langle b \vert 1 \rangle \Vert^2 + \Vert \langle b \vert 0 \rangle \Vert^2 \right)}{\Vert \langle b \vert 1 \rangle \Vert^2 + \Vert \langle b \vert 0 \rangle \Vert^2} = E_1.
    \]

    \item 由于存在$n$个约束条件$\langle b_1 \vert ~~ \rangle = \langle b_2 \vert ~~ \rangle = \cdots = \langle b_n \vert ~~ \rangle = 0$,结合第一问分析可知,要使得$F(b)$最小,$\ket{~~}$应处于$\ket{0},\ket{1},\cdots,\ket{n}$张成的子空间中。则$\ket{~~}$可以写成
    \[
        \ket{~~} = \sum_{i=0}^n c_i \ket{i}
    \]
    其中$c_i = \langle i \vert ~~ \rangle$。则可以计算$F(b)$如下:
    \[
        F(b) = \frac{\sum_{i=0}^n \vert c_i \vert^2 E_i}{\sum_{i=0}^n \vert c_i \vert^2} \leq \frac{E_n\sum_{i=0}^n \vert c_i \vert^2}{\sum_{i=0}^n \vert c_i \vert^2} = E_n.
    \]
\end{enumerate}
\end{sol}

\begin{prob}[]接上题
    \vspace{-0.5em}
    \begin{enumerate}[(i)]
        \item 如上题,假设一约束条件C施加于所有的态矢量。所有的本征值和本征态相应的改变,即$E_n\to E_n', \ket{n} \to \ket{n'}$。利用最大-最小原理,证明:$E_0 \leq E_0', E_1 \leq E_1', \cdots, E_n \leq E_n', \cdots$。
        \item 考虑一个具有固定边界B薄膜的运动。其特征频率由满足边界条件$\varphi = 0$的方程$-\nabla^2\varphi = \omega_n^2\varphi$确定。令$0 < \omega_0 \leq \omega_1 \leq \omega_2 \leq \cdots$。如果施加以约束条件使得振幅$\varphi$在一闭曲线$C$内为零,则特征频率$\omega_n$变为$\omega_n'$。证明$\omega_n \leq \omega_n'$。
    \end{enumerate}
\end{prob}
\begin{sol}
\begin{enumerate}[(i)]
    \item 首先考虑$\hat{H}$本征态空间为有限维,本征值分别为$E_0,E_1, \cdots, E_{l-1}$。由最大最小原理可知:对于任意态矢量$\ket{b}$,$F(b)$为$\frac{\bra{~~} \hat{H} \ket{~~}}{\langle ~~ \vert ~~ \rangle}$满足$\langle b \vert ~~ \rangle = 0$的最大值,则$F(b)$的最小值为$E_{l-2}$。 
    
    不妨令约束条件表示为$\langle b \vert ~~ \rangle = 0$,则有$E_{l-1}' \geq E_{l-2}$,新的本征态计作$\ket{\phi_{l-1}'}$

    约束条件变为$\langle b \vert ~~ \rangle = \langle \phi_{l-1}' \vert ~~ \rangle = 0$,则有$E_{l-1}' \geq E_{l-3}$新的本征态计作$\ket{\phi_{l-2}'}$。以此类推,由于约束条件的添加,本征态空间从l维变成l-1维。对本征值变标号则有:
    \[
        E_0'\geq E_0, E_1' \geq E_1, \cdots, E_{l-2}' \geq E_{l-2}.
    \]
    推广至无穷维则有:
    \[
        E_0'\geq E_0, E_1' \geq E_1, \cdots, E_{l-2}' \geq E_{l-2},\cdots
    \]

    \item 若要求在一闭曲线内为0,即施加了一约束条件C,由(i)问可知,新本征值$\omega_n'$满足$\omega_n'\geq \omega_n$。
\end{enumerate}
\end{sol}

\end{document}